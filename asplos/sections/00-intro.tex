\section{Introduction}\label{sec:intro}
Given the complexity of modern electronic systems, a high degree of automation
is required to develop custom hardware within sensible timelines. At the
highest level, FPGA and ASIC design flows can be split into logical synthesis
and physical synthesis (optimize timing, placement and routing, etc..). This
division of work produces suboptimal designs, and neither are the invididual
synthesis steps locally optimal on their own. However, logic minimization
problems in general are NP-Hard~\todo{cite}, and modern EDA flows bring compile
times down to the human timescale while maintaining acceptable QoR (quality of
results).

With the end of Moore's Law scaling, chip area becomes a tighter constraint,
and logic synthesis is more of a bottleneck. Hence, future synthesis tools will
need to expand the design spaces they explore and find more optimal solutions.
Nonetheless, finding provably optimal circuits is computationally intractable.
In this paper, we will introduce how FPGA technology mapping can be augmented
with e-graph data structures to find \textit{more} exact solutions, without
significantly increasing compile times.

Technology mapping is the hand-off between logical synthesis and physical
synthesis. It converts digital logic written in RTL into a netlist of gates
that belong to the target cell library. For FPGAs, the primary target cell is
the LUT (lookup table). Since every $k$-LUT can be re-programmed to satisfy any
$k$ input boolean function, FPGA technology mapping has a uniquely large
solution space. \todo{It looks like a node covering problem}