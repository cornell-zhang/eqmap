\section{E-Graph Construction}\label{sec:rewrites}
\todo{intro the section}
\todo{fix rule counts, reorder sections}
\subsection{Simplifying Degenerate LUTs}\label{sec:rewrites:degen}

\textbf{Definition:} A LUT's configuration $F : \Bk \rightarrow \B$ is \textit{degenerate} if there exists a Shannon expansion $F = x_i \cdot F_{x_i} + \overline{x_i} \cdot F_{\overline{x}_i}$
such that $F_{x_i} = F_{\overline{x}_i}$ for some $i \in \{ 0, \ldots, k -1\}$. In other words, $F = F_{x_i} = F_{\overline{x}_i}$.

The output of a degenerate LUT is not dependent on one of its inputs. Hence, it
can be rewritten into a LUT which uses fewer inputs. This rule is applied by
computing the Shannon expansions of LUTs and checking for equivalence. For
$k=3$, the rules takes on the following form:

\begin{verbatim}
    (LUT F x0 x1 x2) => (LUT F' x0 x1)
        if F(x0, x1, false) == F(x0, x1, true)
        where F'(x0, x1) := F(x0, x1, true)
\end{verbatim}

One rule is instantiated for each LUT size $k =1$ through 6. One should notice
that LUTs which are constant functions are also handled by this rule. Since
this rule is computationally expensive, it is applied greedily as a
pre-processing step before the e-graph is built. None of the other rewrite
rules create degenerate LUTs, so this has no impact on results. Of course this
rule can be enabled at any time, if it were necessary.

\subsection{Partial Application}\label{sec:rewrites:application}
A LUT with a constant input can be partially evaluated to a LUT with one less
input. This rule is similar to the last. It computes the Shannon expansion
along the constant variable and chooses the cofactor that matches the state of
the constant input. Applying this rule greedily in combination with the
previous one is equivalent to constant propagation. As an example, the
pseudocode for $k=3$ is written as follows:

\begin{verbatim}
    (LUT F x0 x1 false) => (LUT F' x0 x1)
        where F'(x0, x1) := F(x0, x1, false)
\end{verbatim}

\subsection{Functional Composition}\label{sec:rewrites:composition}

Cascaded LUTs can be packed into a single LUT, as long as the size of the cut
of logic has at most 6 leaf nodes. For instance, a circuit that implements
$F(x_0, x_1, G(x_2, x_3))$ with a 3-LUT and 2-LUT can be rewritten as a 4-LUT
that implements some $H(x_0, x_1, x_2, x_3)$. In pseudocode, this would take on
the following form:
\begin{verbatim}
    (LUT F x0 x1 (LUT G x2 x3)) => (LUT H x0 x1 x2 x3)
        where H(x0, x1, x2, x3) := F(x0, x1, G(x2, x3))
\end{verbatim}

The search patterns \texttt{x0} and \texttt{x1} can match any node. They are
not necessarily principal inputs, and hence can be outputs from other LUTs. As
a consequence, this rule can be chained together many times in varying orders
to pack a sub-circuit into a single LUT. As demonstrated in the next section,
we can write this rule for one specific input position, without loss of
generality. Therefore, we only need to sweep over the size of the two LUTs in
the search pattern. In total, there are $6*6 = 36$ LUT packing rules. When the
cut of logic is larger than 6 leaves, the rules fail gracefully and do not
interefere with reaching equality saturation.
\subsection{LUT Symmetries}\label{sec:rewrites:symmetry}

The semantics of LUTs should not depend on the order of their inputs. If two
LUTs have permuted inputs but are otherwise functionally identical, they should
belong to the same e-class in the graph. That is, \mbox{\texttt{(LUT F .. xi ..
        xj ..)}} is semantically equivalent to \mbox{\texttt{(LUT G .. xj .. xi ..)}}
if and only if $G = F \odot \sigma^{-1}$, where $\sigma \in S_k$ is the
permutation applied to the inputs.

\begin{proof}
    $\odot$ is a right-action defined for the sake of permuting the inputs to a function before they are applied:
    \[ \odot : \big (\Bk \rightarrow \mathbb{Z}_2 \big ) \times S_k \rightarrow \big (\Bk \rightarrow \mathbb{Z}_2 \big )\]
    \[ F \odot \sigma : (x_0, x_1, \ldots, x_{k-1}) \mapsto F(x_{\sigma(0)}, x_{\sigma(1)}, \ldots, x_{\sigma(k-1)}) \]

    It is trivial to prove that this right-action is associative:
    \begin{align*}
        (F \odot \sigma_1) \odot \sigma_2 & = F(x_{\sigma_2(\sigma_1(0))}, x_{\sigma_2(\sigma_1(1))}, \ldots, x_{\sigma_2(\sigma_1(k-1))}) \\
        (F \odot \sigma_1) \odot \sigma_2 & = F \odot (\sigma_2 \circ \sigma_1)
    \end{align*}
    With this property, the rest follows directly:
    \[ F = G \odot
        \sigma \iff F \odot \sigma^{-1} = (G \odot \sigma) \odot \sigma^{-1} = G \]
\end{proof}

Therefore, we can conclude that $k$-LUTs have as much symmetry as can be
generated by the group $S_k$. This formal approach may be considered overkill,
but it has two major consequences. First, it precisely reveals how many e-graph
rewrite rules are needed to generate all the symmetries of a LUT. For any
$k$-LUT with program $F$, we need exactly as many rules as it takes to generate
$F \odot S_k$. It is a well-known fact in algebra that that the $k-1$ adjacent
transpositions generate $S_k$~\cite{sgroup}. Therefore, we can insert an
e-graph rewrite rule for each adjacent transposition. In total, this is
$\sum_{k=2}^{6} (k-1) = 15$ rules to encapsulate symmetry for every LUT size.
The second consequence is that every other rewrite rule can now be defined for
one input position, without loss of generality. This reduces the total number
of rewrite rules, making it easier to rationalize about the rule system and
which types of optimizations are reachable.

\subsection{LUTs with Domain Restrictions}\label{sec:rewrites:restrict}

\textbf{Definition:} A lookup table \texttt{(LUT F x0 x1 ...)} is \textit{restricted} if \texttt{xi == xj} for some $ i, j \in \{0, \ldots, k-1\}, \; i \neq j$.
In other words, the domain of the LUT is restricted.

The main advantage of using e-graphs is the compact way in which it represents
notions of equality. When considering the entire set of rewrite rules under
composition, we can observe new equalities being formed in the e-graph.
Whenever an equality is found between two of the inputs to a $k$-LUT, it can be
rewritten with a $(k-1)$-LUT. We simply need to define and compute
$\texttt{restrict(F, i, j)}$ which maps $F : \Bx{k} \rightarrow \B$ to the
domain-restricted $F \vert_{x_i = x_j} : \Bx{k-1} \rightarrow \B$. In
pseudocode, the rewrite rule can be rwritten as follows:

\begin{verbatim}
    (LUT F x0 x1 x1) => (LUT G x0 x1)
        where G := restrict(F, 1, 2)
\end{verbatim}

Since e-graph rewrite patterns search on e-classes, this rule is automatically
re-checked when e-classes are merged. Since LUT symmetry is represented in the
graph, only one rule is needed for each lut size $k=2$ through 6.

\subsection{Functional Decomposition}\label{sec:rewrites:decomp}

Decomposing boolean functions and logic minimization in general is
NP-complete~\cite{logicmin}. Correspondingly, decomposing LUTs explodes the
size and build time of the e-graph. However, we can still define rewrites that
look for fully disjoint decompositions in one or more variables. This rule has
no structural element to search for, so it runs every time an e-class is
updated. Our implementation computes the Shannon expansion of a $k$-LUT's
function $F$ and checks that both cofactors are cognates in a loose sense. For
instance, given $k=3$ then it is true that:

\begin{align*}
    F(x_0, x_1, x_2) = G(x_0, H(x_1, x_2)) \iff & F_{x_0} (x_1, x_2) = G_{x_0} (H(x_1, x_2))                             \\
                                                & \land F_{\overline{x}_0} (x_1, x_2) = G_{\overline{x}_0} (H(x_1, x_2))
\end{align*}

In practice, our implementation checks if either of the cofactors are constant
functions or if the cofactors are equivalent up to complementation.

\subsection{Register Retiming}\label{sec:rewrites:retiming}
Register retiming, as a purely structural rule, can be implemented with a
simple search and apply pattern. An example for $k=3$ would be written as
follows:

\begin{verbatim}
    (LUT F (REG x0) (REG x1)) <=> (REG (LUT F x0 x1))
\end{verbatim}

Unlike the other rules, this rule is searched for in both directions.