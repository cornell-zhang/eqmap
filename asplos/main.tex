% use the base acmart.cls
% use the sigplan proceeding template with the default 10 pt fonts
% nonacm option removes ACM related text in the submission. 
\documentclass[sigplan,nonacm]{acmart}

\newcommand{\fixme}[1]{\textcolor{red}{#1}}
\newcommand{\todo}[1]{\textcolor{red}{TODO: #1}}
\newcommand{\fullname}{Lut-Synth}
\newcommand{\shortname}{SLS}

% enable page numbers
\settopmatter{printfolios=true}


\begin{document}

\title{\fullname}

\begin{abstract}
    FPGA technology mapping is a well-studied problem and has been an area of interest in EDA tool design for decades.
    In most respects, the computational complexity of technology mapping is understood, and adequate heuristic algorithms have been sucessfully employed to mitigate compile times while maintaing high QoR (quality of results).
    As transistor scaling comes to an end within the next few years, synthesis tools will become more of the bottleneck in designing high performance accelerators. As a solution, we introduce \shortname{}, an e-graph driven technology mapper that can better span the cavern between SAT-based exact synthesis and heuristic cut enumeration techniques.
    In our experiments, we show that \shortname{} can synthesize circuits with \todo{X\%} fewer LUTs on average, \textit{without ever increasing circuit depth}.
    We also provide an empirical analysis the runtime of \shortname{} and show that it is still practical for large (~10,000 LUT) designs.
    Finally, we demonstrate that our compiler infrastructure is reusable, and future work can use our compiler for RTL equivalence checking or auditing the QoR of synthesis tools.
\end{abstract}
\maketitle % should come after the abstract

\section{Introduction}\label{sec:intro}
Given the complexity of modern electronic systems, a high degree of automation
is required to develop custom hardware within sensible timelines. At the
highest level, FPGA and ASIC design flows can be split into logic synthesis and
physical design (e.g., floorplanning placement and routing). This division of
work produces suboptimal designs, and neither are the individual synthesis
steps locally optimal on their own. However, circuit minimization problems in
general are NP-Hard~\cite{logicmin,twolevellogic}, and modern EDA flows bring
compile times down to the human timescale while maintaining acceptable quality
of results (QoR).

With the end of Dennard scaling and Moore's Law fading, the quality of logic
synthesis becomes more important. Hence, future synthesis tools will need to
expand the design spaces they explore and find more optimal solutions. Still,
finding provably optimal circuits is computationally intractable. In this
paper, we introduce how FPGA technology mapping can be augmented with e-graph
data structures to find \textit{more} exact solutions, without significantly
increasing compile times.

Technology mapping is the hand-off between logic synthesis and physical design.
It converts the abstract Boolean logic into a network of circuit elements that
belong to the target cell library. For FPGAs, the primary target cell is the
lookup table (LUT). Since every $k$-LUT can be re-programmed to satisfy any $k$
input boolean function, FPGA technology mapping has an unmistakably large
solution space. Whether the circuit is optimized for latency or area, most FPGA
tools approach technology mapping as a graph covering problem~\cite{flowmap,
    daomap, attmap, imap}. In the literature, a group of circuit nodes implemented
by a $k$-LUT is called a $k$-feasible cut of logic, and the generation of all
cuts is called cut enumeration. These structural mapping techniques rely on the
topology of the input circuit, and hence they are prone to \textit{structural
    bias}.

In contrast, functional mappers attempt to decompose the Boolean functionality
into smaller sub-functions which can be realized by $k$-LUTs. Such mappers are
a more exact approach, and often use SAT solvers to drive
synthesis~\cite{satmap,satmap2}. Other works employ Boolean matching to speedup
of technology mapping by identifying known Boolean
structures~\cite{boolmatch,fastboolmatch}. However, exact synthesis tools
cannot be scaled past tens of gates. As a consequence, cut enumeration and
functional mapping lie on two different extremes. The former is fast but
limited by the input structure, while the latter is unbiased but fundamentally
unscalable.

For this reason, we propose an e-graph driven technology mapper than can better
span the time-QoR spectrum. Equality graphs, referred to as e-graphs, are a
data structure which use union-find operations to compactly represent abstract
equivalence relations~\cite{eggpaper}. Whereas typical optimizing compilers
apply a greedy sequence of transformation passes, e-graphs rewrite terms
iteratively in a nondestructive fashion. Our work seeks to evaluate the
suitability of e-graphs for logic synthesis, specifically for technology
mapping to FPGAs. By using the output mappings of RTL synthesis tools as an
initial solution, we can use e-graphs to incrementally explore more compact
circuit topologies.

To that end, we propose \shortname{}: a tool for repacking FPGA netlists into
more compact forms---without increasing circuit depth. Our results show many
benchmarks, big and small, which synthesize to significantly fewer LUTs over
vendor EDA tools. To that end, our work makes the following contributions:

\begin{itemize}
    \item We formulate an intermediate language and set of e-graph rewrite rules that can
          explore circuit topologies that heuristic approaches miss.
    \item We evaluate our compiler against \nbenchmarks{} benchmarks combined from three
          sources: EPFL~\cite{epflbench}, ISCAS'85~\cite{iscasbench}, and
          LGSynth'91~\cite{lgsynthbench}. The results show improvements in LUT count
          without significant increases to compile time.
    \item Finally, \shortname{} is packaged as a Verilog-to-Verilog tool that can be
          dropped into existing RTL flows.
\end{itemize}

Before elaborating on our methodology and experimental setup, we first discuss
related ideas in technology mapping and e-graph driven compilers. Then, the
results section illustrates the typical reduction in LUT count our tool
achieves without increasing circuit depth. Lastly, we discuss the future work
of our compiler.
\section{E-Graph Construction}\label{sec:rewrites}

In the follow sections, we call the truth table of a LUT the \textit{program}
or \textit{function} interchangeably. The program of a $k$-LUT is modeled as a
function $F : \Bk \rightarrow \B$, where $\mathbb{Z}_2 = \mathbb{Z}/2\mathbb{Z}
    = \mathbb{F}_2 = \{0,1\}$. To that end, we can endow the grammar of the netlist
with precise denotational semantics. Formalizing the meaning of Verilog
netlists in a functional way is critical to ensuring correct rewrite rules and
further understanding of the high-level interactions of these rules.

\subsection{Simplifying Degenerate LUTs}\label{sec:rewrites:degen}

\textbf{Definition:} A LUT's configuration $F : \Bk \rightarrow \B$ is \textit{degenerate} if there exists a Shannon expansion $F = x_i \cdot F_{x_i} + \overline{x_i} \cdot F_{\overline{x}_i}$
such that $F_{x_i} = F_{\overline{x}_i}$ for some $i \in \{ 0, \ldots, k -1\}$. In other words, $F = F_{x_i} = F_{\overline{x}_i}$.

The output of a degenerate LUT is not dependent on one of its inputs. Hence, it
can be rewritten into a LUT which uses fewer inputs. This rule is applied by
computing the Shannon expansions of LUTs and checking for equivalence. For
$k=3$, the rules takes on the following form:

\begin{verbatim}
    (LUT F x0 x1 x2) => (LUT F' x0 x1)
        if F(x0, x1, false) == F(x0, x1, true)
        where F'(x0, x1) := F(x0, x1, true)
\end{verbatim}

One rule is instantiated for each LUT size $k =1$ through 6. One should notice
that LUTs which are constant functions are also handled by this rule. Since
this rule is computationally expensive, it is applied greedily as a
pre-processing step before the e-graph is built. None of the other rewrite
rules create degenerate LUTs, so this has no impact on results. Of course this
rule can be enabled at any time, if it were necessary.

\subsection{Partial Application}\label{sec:rewrites:application}
A LUT with a constant input can be partially evaluated to a LUT with one less
input. This rule is similar to the last. It computes the Shannon expansion
along the constant variable and chooses the cofactor that matches the state of
the constant input. Applying this rule greedily in combination with the
previous one is equivalent to constant propagation. As an example, the
pseudocode for $k=3$ is written as follows:

\begin{verbatim}
    (LUT F x0 x1 false) => (LUT F' x0 x1)
        where F'(x0, x1) := F(x0, x1, false)
\end{verbatim}

\subsection{Functional Composition}\label{sec:rewrites:composition}

Cascaded LUTs can be packed into a single LUT, as long as the size of the cut
of logic has at most 6 leaf nodes. For instance, a circuit that implements
$F(x_0, x_1, G(x_2, x_3))$ with a 3-LUT and 2-LUT can be rewritten as a 4-LUT
that implements some $H(x_0, x_1, x_2, x_3)$. In pseudocode, this would take on
the following form:
\begin{verbatim}
    (LUT F x0 x1 (LUT G x2 x3)) => (LUT H x0 x1 x2 x3)
        where H(x0, x1, x2, x3) := F(x0, x1, G(x2, x3))
\end{verbatim}

The search patterns \texttt{x0} and \texttt{x1} can match any node. They are
not necessarily principal inputs, and hence can be outputs from other LUTs. As
a consequence, this rule can be chained together many times in varying orders
to pack a sub-circuit into a single LUT. As demonstrated in the next section,
we can write this rule for one specific input position, without loss of
generality. Therefore, we only need to sweep over the size of the two LUTs in
the search pattern. In total, there are $6*6 = 36$ LUT packing rules. When the
cut of logic is larger than 6 leaves, the rules fail gracefully and do not
interefere with reaching equality saturation.
\subsection{LUT Symmetries}\label{sec:rewrites:symmetry}

The semantics of LUTs should not depend on the order of their inputs. If two
LUTs have permuted inputs but are otherwise functionally identical, they should
belong to the same e-class in the graph. That is, \mbox{\texttt{(LUT F .. xi ..
        xj ..)}} is semantically equivalent to \mbox{\texttt{(LUT G .. xj .. xi ..)}}
if and only if $G = F \odot \sigma^{-1}$, where $\sigma \in S_k$ is the
permutation applied to the inputs.

\begin{proof}
    $\odot$ is a right-action defined for the sake of permuting the inputs to a function before they are applied:
    \[ \odot : \big (\Bk \rightarrow \mathbb{Z}_2 \big ) \times S_k \rightarrow \big (\Bk \rightarrow \mathbb{Z}_2 \big )\]
    \[ F \odot \sigma : (x_0, x_1, \ldots, x_{k-1}) \mapsto F(x_{\sigma(0)}, x_{\sigma(1)}, \ldots, x_{\sigma(k-1)}) \]

    It is trivial to prove that this right-action is associative:
    \begin{align*}
        (F \odot \sigma_1) \odot \sigma_2 & = F(x_{\sigma_2(\sigma_1(0))}, x_{\sigma_2(\sigma_1(1))}, \ldots, x_{\sigma_2(\sigma_1(k-1))}) \\
        (F \odot \sigma_1) \odot \sigma_2 & = F \odot (\sigma_2 \circ \sigma_1)
    \end{align*}
    With this property, the rest follows directly:
    \[ F = G \odot
        \sigma \iff F \odot \sigma^{-1} = (G \odot \sigma) \odot \sigma^{-1} = G \]
\end{proof}

Therefore, we can conclude that $k$-LUTs have as much symmetry as can be
generated by the group $S_k$. This formal approach may be considered overkill,
but it has two major consequences. First, it precisely reveals how many e-graph
rewrite rules are needed to generate all the symmetries of a LUT. For any
$k$-LUT with program $F$, we need exactly as many rules as it takes to generate
$F \odot S_k$. It is a well-known fact in algebra that that the $k-1$ adjacent
transpositions generate $S_k$~\cite{sgroup}. Therefore, we can insert an
e-graph rewrite rule for each adjacent transposition. In total, this is
$\sum_{k=2}^{6} (k-1) = 15$ rules to encapsulate symmetry for every LUT size.
The second consequence is that every other rewrite rule can now be defined for
one input position, without loss of generality. This reduces the total number
of rewrite rules, making it easier to rationalize about the rule system and
which types of optimizations are reachable.

\subsection{LUTs with Domain Restrictions}\label{sec:rewrites:restrict}

\textbf{Definition:} A lookup table \texttt{(LUT F x0 x1 ...)} is \textit{restricted} if \texttt{xi == xj} for some $ i, j \in \{0, \ldots, k-1\}, \; i \neq j$.
In other words, the domain of the LUT is restricted.

The main advantage of using e-graphs is the compact way in which it represents
notions of equality. When considering the entire set of rewrite rules under
composition, we can observe new equalities being formed in the e-graph.
Whenever an equality is found between two of the inputs to a $k$-LUT, it can be
rewritten with a $(k-1)$-LUT. We simply need to define and compute
$\texttt{restrict(F, i, j)}$ which maps $F : \Bx{k} \rightarrow \B$ to the
domain-restricted $F \vert_{x_i = x_j} : \Bx{k-1} \rightarrow \B$. In
pseudocode, the rewrite rule can be rwritten as follows:

\begin{verbatim}
    (LUT F x0 x1 x1) => (LUT G x0 x1)
        where G := restrict(F, 1, 2)
\end{verbatim}

Since e-graph rewrite patterns search on e-classes, this rule is automatically
re-checked when e-classes are merged. Since LUT symmetry is represented in the
graph, only one rule is needed for each lut size $k=2$ through 6.

\subsection{Functional Decomposition}\label{sec:rewrites:decomp}

Decomposing boolean functions and logic minimization in general is
NP-complete~\cite{logicmin}. Correspondingly, decomposing LUTs explodes the
size and build time of the e-graph. However, we can still define rewrites that
look for fully disjoint decompositions in one or more variables. This rule has
no structural element to search for, so it runs every time an e-class is
updated. Our implementation computes the Shannon expansion of a $k$-LUT's
function $F$ and checks that both cofactors are cognates in a loose sense. For
instance, given $k=3$ then it is true that:

\begin{align*}
    F(x_0, x_1, x_2) = G(x_0, H(x_1, x_2)) \iff & F_{x_0} (x_1, x_2) = G_{x_0} (H(x_1, x_2))                             \\
                                                & \land F_{\overline{x}_0} (x_1, x_2) = G_{\overline{x}_0} (H(x_1, x_2))
\end{align*}

In practice, our implementation checks if either of the cofactors are constant
functions or if the cofactors are equivalent up to complementation.

\subsection{Register Retiming}\label{sec:rewrites:retiming}
Register retiming, as a purely structural rule, can be implemented with a
simple search and apply pattern. An example for $k=3$ would be written as
follows:

\begin{verbatim}
    (LUT F (REG x0) (REG x1)) <=> (REG (LUT F x0 x1))
\end{verbatim}

Unlike the other rules, this rule is searched for in both directions.

% Possible outline:
% 1. Related work
% 2. Egraph Construction
% 2.a. Simplifying degenerate LUTs
% 2.b. Functional composition
% 2.c. Functional decomposition
% 2.d. Symmetries of non-degenerate LUTs
% 2.e LUTs under restriction
% Experimental Setup
% results
% Future work

% use the ACM bibliography style
\bibliographystyle{ACM-Reference-Format}
\bibliography{references}

% \newpage
%%
%% If your work has an appendix, this is the place to put it.
% \appendix

\end{document}