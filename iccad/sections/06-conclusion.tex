\section{Conclusion and Future Work}\label{sec:conclusion}
While technology mapping has been studied for decades, the end of transistor
scaling will require new logic synthesis tools that are less heuristic in
nature. This work seeks to demonstrate that there are practical solutions that
can bridge the gap between SAT-based exact synthesis and cut enumeration
techniques. With \shortname{}, we can use e-graphs to improve FPGA technology
mapping with a post-processing compilation step. Specifically, our results show
that \shortname{} can synthesize circuits with \metric{} on average without
increasing circuit depth. While other techniques may approach or beat these
gains, equality saturation as a formal method makes e-graphs a particularly
trustworthy method by which to transform circuits. Lastly, e-graph construction
is decoupled from extraction, meaning this type of flow is particularly
adaptive to new optimization objectives.

As for future work, the main research problem is the development and
integration of more sophisticated extraction techniques. After all, e-graph
construction defers all the choices in circuit topology. The QoR is still
completely contingent on e-graph extraction. While simple greedy extraction
clearly serves a purpose, it fails to capture the full potential of
\shortname{}'s rewriting system.