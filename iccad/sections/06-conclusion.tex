\section{Conclusion and Future Work}\label{sec:conclusion}
While technology mapping has been studied for decades, the end of transistor
scaling will require new logic synthesis tools that are less heuristic in
nature. This work seeks to demonstrate that there are practical solutions that
can bridge the gap between SAT-based exact synthesis and cut enumeration
techniques. With \shortname{}, we can use e-graphs to improve FPGA technology
mapping with a post-processing compilation step. Specifically, our results show
that \shortname{} can synthesize circuits with \metric{} on average without
increasing circuit depth. While other techniques may approach or beat these
gains, equality saturation as a formal method makes e-graphs a particularly
trustworthy method by which to transform circuits. Lastly, e-graph construction
is decoupled from extraction, meaning this type of flow is particularly
adaptive to new optimization objectives.

As for future work, the main research problem is the integration of more
sophisticated extraction techniques. The final QoR is still largely contingent
on e-graph extraction, and more advanced extraction techniques would likely
improve the results even further. While simple greedy extraction clearly serves
a purpose, it fails to capture the full potential of \shortname{}'s rewriting
system. Nonetheless, \shortname{} is a promising step towards EDA tools that
can better span the gap between exact and fully heuristic logic synthesis.
