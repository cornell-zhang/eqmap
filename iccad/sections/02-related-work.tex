\section{Related Work}\label{sec:relatedwork}
% Related work todo
% - dsd
\subsection{FPGA Technology Mapping}\label{sec:relatedwork:fpga}
FPGA technology mapping converts abstract RTL logic into a netlist of lookup
tables (LUTs). Due to the computational complexity of this problem, most
implementations avoid restructuring the input logic and are essentially graph
covering algorithms. Where competing implementations vary, however, is the set
of heuristics they use to guide optimization. Broadly speaking, FPGA LUT
packing can be divided into architecture-specific and architecture-nonspecific
optimizations. Architecture-specific LUT packing encapsulates optimizations
that reduce routing congestion by more efficiently using intra-CLB routing
resources. Examples include mapping dual-output functions to fractured
LUTs~\cite{fraclut} or using dedicated multiplexers to implement functions with
more than 6 inputs~\cite{ug574}.

In contrast, architecture-nonspecific LUT packing is more general as it
attempts to mitigate the structural bias of the input logic. As an example,
AGDMap~\cite{adaptdecomp} decomposes simple logic gates with large fanin to
enable the exploration of better graph coverings. However, structural bias can
take on many forms. For instance, FlowMap~\cite{flowmap} cites that the
non-monotone clustering of logic is the fundamental difficulty of LUT-based
technology mapping. In other words, a $k$-feasible cut of logic may contain
subcuts that are \textit{not} $k$-feasible. Overcoming bouts of non-monotone
clustering requires a more elaborate cut-selection algorithm, and our work lays
the foundation for a formal, reasoning-based approach to the problem.

\subsection{E-Graph Superoptimization}\label{sec:relatedwork:egraph}
Equivalence graphs (e-graphs) are a data structure originally conceived to
facilitate automated proof generation~\cite{eggpaper, eqsat}. For example,
e-graphs can be used to rewrite mathematical expressions~\cite{egraphmath} or
for automated reasoning about functional programs~\cite{cclemma}. In recent
years, e-graphs and equality saturation have enjoyed renewed popularity within
the compilers field. Equality saturation is useful for optimizing compilers,
because it defers greedy program transformations. Extracting solutions from
saturated e-graphs can result in more optimal---sometimes provably
optimal---programs. In contrast, traditional compilers use a pass pipeline
architecture which suffers from a \textit{phase-ordering problem}. In other
words, there is never an ordering of transformation passes that is optimal for
all input programs. This is a deep-rooted issue in compiler design, but the
problem is particularly consequential for hardware design. Several recent works
use e-graphs to improve upon typical optimizing compiler architecture.

SEER~\cite{seer} uses e-graphs to optimize the data-level and pipeline
parallelism of control flow in high-level synthesis (HLS) programs.
IMpress~\cite{impress} also uses e-graphs at the HLS level, optimizing the
muliplication of large bit-width integers. At a lower level,
ROVER~\cite{rover,roverbl,egraphconstraints} rewrites arithmetic data paths at
the word and bit levels. ROVER straddles the RTL and physical level of
abstraction, making it more general purpose than IMpredss. In any case, the
work which is most similar in its goals to ours is E-Syn~\cite{esynth}. E-Syn
rewrites the Boolean algebra of a circuit using known properties like De
Morgan's laws and the consensus theorem. Ultimately, E-Syn performs its
optimizations during technology-independent synthesis steps, whereas
\shortname{} applies as a post-processing step \textit{after} technology
mapping. Furthermore, \shortname{}'s rewriting system models the netlist in
terms of total functions, rather than as expressions over a Boolean algebra.