\section{Introduction}\label{sec:intro}
Given the complexity of modern electronic systems, a high degree of automation
is required to develop custom hardware within sensible timelines. At the
highest level, FPGA and ASIC design flows can be split into logic synthesis and
physical design (e.g., floorplanning placement and routing). This division of
work produces suboptimal designs, and neither are the individual synthesis
steps locally optimal on their own. However, circuit minimization problems in
general are NP-Hard~\cite{logicmin,twolevellogic}, and modern EDA flows bring
compile times down to the human timescale while maintaining acceptable quality
of results (QoR).

With the end of Dennard scaling and Moore's Law fading, the quality of logic
synthesis becomes more important. Hence, future synthesis tools will need to
expand the design spaces they explore and find more optimal solutions. Still,
finding provably optimal circuits is computationally intractable. In this
paper, we introduce how FPGA technology mapping can be augmented with e-graph
data structures to find \textit{more} exact solutions, without significantly
increasing compile times.

Technology mapping is the hand-off between logic synthesis and physical design.
It converts the abstract Boolean logic into a network of circuit elements that
belong to the target cell library. For FPGAs, the primary target cell is the
lookup table (LUT). Since every $k$-LUT can be re-programmed to satisfy any $k$
input boolean function, FPGA technology mapping has an unmistakably large
solution space. Whether the circuit is optimized for latency or area, most FPGA
tools approach technology mapping as a graph covering problem~\cite{flowmap,
    daomap, attmap, imap}. In the literature, a group of circuit nodes implemented
by a $k$-LUT is called a $k$-feasible cut of logic, and the generation of all
cuts is called cut enumeration. These structural mapping techniques rely on the
topology of the input circuit, and hence they are prone to \textit{structural
    bias}.

In contrast, functional mappers attempt to decompose the Boolean functionality
into smaller sub-functions which can be realized by $k$-LUTs. Such mappers are
a more exact approach, and often use SAT solvers to drive
synthesis~\cite{satmap,satmap2}. Other works employ Boolean matching to speedup
of technology mapping by identifying known Boolean
structures~\cite{boolmatch,fastboolmatch}. However, exact synthesis tools
cannot be scaled past tens of gates. As a consequence, cut enumeration and
functional mapping lie on two different extremes. The former is fast but
limited by the input structure, while the latter is unbiased but fundamentally
unscalable.

For this reason, we propose an e-graph driven technology mapper than can better
span the time-QoR spectrum. Equality graphs, referred to as e-graphs, are a
data structure which use union-find operations to compactly represent abstract
equivalence relations~\cite{eggpaper}. Whereas typical optimizing compilers
apply a greedy sequence of transformation passes, e-graphs rewrite terms
iteratively in a nondestructive fashion. Our work seeks to evaluate the
suitability of e-graphs for logic synthesis, specifically for technology
mapping to FPGAs. By using the output mappings of RTL synthesis tools as an
initial solution, we can use e-graphs to incrementally explore more compact
circuit topologies.

To that end, we propose \shortname{}: a tool for repacking FPGA netlists into
more compact forms---without increasing circuit depth. Our results show many
benchmarks, big and small, which synthesize to significantly fewer LUTs over
vendor EDA tools. To that end, our work makes the following contributions:

\begin{itemize}
    \item We formulate an intermediate language and set of e-graph rewrite rules that can
          explore circuit topologies that heuristic approaches miss.
    \item We evaluate our compiler against \nbenchmarks{} benchmarks combined from three
          sources: EPFL~\cite{epflbench}, ISCAS'85~\cite{iscasbench}, and
          LGSynth'91~\cite{lgsynthbench}. The results show improvements in LUT count
          without significant increases to compile time.
    \item Finally, \shortname{} is packaged as a Verilog-to-Verilog tool that can be
          dropped into existing RTL flows.
\end{itemize}

Before elaborating on our methodology and experimental setup, we first discuss
related ideas in technology mapping and e-graph driven compilers. Then, the
results section illustrates the typical reduction in LUT count our tool
achieves without increasing circuit depth. Lastly, we discuss the future work
of our compiler.