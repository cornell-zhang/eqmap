\section{Background}\label{sec:background}
\todo{explain background}
% Logic Synthesis

% Difficulty of the FPGA Mapping problem

\subsection{Formal Properties of E-Graphs}\label{sec:background:egraph}
Equivalence graphs (e-graphs) are a data structure that were originally
conceived to facilitate automated proof generation~\cite{eggpaper, eqsat}. For
example, e-graphs can be used to simplify mathematical
expressions~\cite{egraphmath} or for automated reasoning about functional
programs~\cite{cclemma}.\todo{this section}