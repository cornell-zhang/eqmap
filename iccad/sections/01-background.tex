\section{Background}\label{sec:background}
\todo{work in a motivating example for LUT packing and e-graphs to do the rewriting}
% Logic Synthesis

\subsection{FPGA Technology Mapping}\label{sec:background:fpga}
FPGA technology mapping converts abstract RTL logic into a netlist of lookup
tables (LUTs). Due to the computational complexity of LUT packing, most
implementations avoid restructuring the input logic and are essentially graph
covering algorithms. Where competing implementations vary, however, is the set
of heuristics they use to guide optimization.

\subsection{Equivalence Graphs}\label{sec:background:egraph}
Equivalence graphs (e-graphs) are a data structure originally conceived to
facilitate automated proof generation~\cite{eggpaper, eqsat}. For example,
e-graphs can be used to rewrite mathematical expressions~\cite{egraphmath} or
for automated reasoning about functional programs~\cite{cclemma}.

Equality saturation is useful for optimizing compilers, because it defers
greedy program transformations. Extracting solutions from saturated e-graphs
can result in more optimal---sometimes provably optimal---programs. In
contrast, traditional compilers use a pass pipeline architecture which suffers
from a \textit{phase-ordering problem}. In other words, there is never an
ordering of transformation passes that is optimal for all input programs. This
is a deep-rooted issue in compiler design, but the problem is particularly
consequential for hardware design.
