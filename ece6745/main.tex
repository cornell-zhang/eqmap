\documentclass[10pt,letterpaper]{article}
\usepackage{fullpage}
\usepackage{xcolor}
\usepackage[top=1in, bottom=1in, left=1in, right=1in]{geometry}

\newcommand{\todo}[1]{\textcolor{red}{TODO:\ #1}}

\title{ECE6745 Project Proposal}
\author{Matthew Hofmann}
\date{\today}

\begin{document}

\maketitle

\section{Introduction}\label{sec:intro}
Lorem ipsum dolor sit amet, consectetur adipiscing elit, sed do eiusmod tempor
incididunt ut labore et dolore magna aliqua. Ut enim ad minim veniam, quis
nostrud exercitation ullamco laboris nisi ut aliquip ex ea commodo consequat.
Duis aute irure dolor in reprehenderit in voluptate velit esse cillum dolore eu
fugiat nulla pariatur. Excepteur sint occaecat cupidatat non proident, sunt in
culpa qui officia deserunt mollit anim id est laborum.

Lorem ipsum dolor sit amet, consectetur adipiscing elit, sed do eiusmod tempor
incididunt ut labore et dolore magna aliqua. Ut enim ad minim veniam, quis
nostrud exercitation ullamco laboris nisi ut aliquip ex ea commodo consequat.
Duis aute irure dolor in reprehenderit in voluptate velit esse cillum dolore eu
fugiat nulla pariatur. Excepteur sint occaecat cupidatat non proident, sunt in
culpa qui officia deserunt mollit anim id est laborum.

Extraction is clear still important.

\section{Background}\label{sec:background}
Equivalence graphs, most commonly referred to as \textit{e-graphs}, are an
automated reasoning tool built around a union-find data
structure~\cite{eggpaper}.

\section{Experiments}\label{sec:experiments}

Each experiment should describe the motivation, the baseline, the methods, and
evaluation technique. As well as the state of the art.

0. Equivalence Checking and Fuzzing

2. Standard cell development

3. Re-synthesis
- For depth
- For area
- flip flop area

The proposal should be maybe 2-3 pages. It should be a step towards the
introduction for the final project report. So you should include a paragraph or
two about the motivation for the project, a paragraph or two that describes
your plan for the baseline and alternative design, a paragraph or two that
describes how you plan to test and evaluate your design, and a few paragraphs
for the literature review. Your annotated bibliography should go at the end.
Note -- you must annotate at least three scholarly references, but you do not
need to annotate every reference. Feel free to include as many references as
you like. You can find project ideas and more information about the literature
search in the Project Ideas handout on the Canvas handouts page.

\bibliographystyle{plain}
\bibliography{references}

\end{document}